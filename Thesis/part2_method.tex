\chapter{Теоретические основы моделирования динамики доходности торговых стратегий}
\section{Моделирование сдвигов}
\note{Модель структурных сдвигов}
Для учета структурных сдвигов с известной точкой перехода можно использовать две различные модели <<до>> и <<после>> \citep{salazar1982}. Этот подход -- естественная идея, кроме того, есть свобода учитывать структурные изменения лишь в тех частях модели, которые в этом нуждаются, как например средние доходности.

\section{Моделирование волатильности}
\note{Обзор методов моделирования волатильности}
Для учета стохастической волатильности используется класс моделей имеющих скрытый марковский процесс \citep{ghahramani2001}:
\begin{itemize}
	\item GARCH \citep{engle1982}
	\item Gaussian Process Volatility Model \citep{han2016}
\end{itemize}
Последний из них интересен тем, что он непараметрический, относительно прост в реализации, встраивается в байесовский подход. Он позволяет крайне гибко моделировать скрытые состояния волатильности.

\note{учет сдвига в моделях волатильности}
Процесс стохастической волатильности будем считать стабильным относительно структурного сдвига. Эта предпосылка требуется для того, чтобы излишне не усложнять модель, так как ее численный вывод достаточно длителен.
\section{Моделирование динамики корреляций}
\note{в чем сложность моделирования динамики корреляций?}
Моделирование динамики корреляций -- сложная задача, это скрытый марковский процесс с большим количеством скрытых состояний, квадратичное по количеству рядов. Без априорных предположений о структуре матрицы корреляций вычислительная сложность достаточно велика.

\note{что сделано на текущий момент?}

\note{что из этого можно использовать}
\section{Байесовский подход к моделированию динамики торговых стратегий}
\note{Зачем тут байесовский подход?}
\note{Как все связать в единой байесовской модели?}
\note{сдвиги абстрагировано}
\note{волатильность абстрагировано}
\note{корреляции абстрагировано}
\section{Структурная модель динамики доходностей торговых стратегий}
\note{граф модель, красивая картинка}
\note{Подход к оценке модели, MCMC}