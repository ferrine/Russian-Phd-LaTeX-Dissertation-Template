\begin{align*}
\mu_\text{is} &\sim p(\mu_\text{is}); \in \Real^k& \text{ожидаемая доходность в обучающем периоде}\\
\mu_\text{oos} &\sim p(\mu_\text{oos});\in \Real^k&\text{ожидаемая доходность в тестовом периоде}\\
\nu &\sim p(\nu);\in \Real^k&\text{ожидаемый логарифм дисперсии}\\
\kappa &\sim p(\kappa);\in \Real& \text{ожидаемая статистика корреляции}\\
\theta_\mathcal{GP} &\sim p(\theta_\mathcal{GP})&\text{параметры многомерного гауссовского процесса}\\
\phi_\mathcal{GP} &\sim p(\phi_\mathcal{GP})&\\
\psi_\mathcal{GP} &\sim p(\psi_\mathcal{GP})&\\
\Delta\mu &\sim \mathcal{GP}(\theta_\mathcal{GP}); f:\Real \to \Real^k&\text{изменение доходности во времени}\\
\Delta\nu &\sim \mathcal{GP}(\phi_\mathcal{GP}); f:\Real \to \Real^k&\text{изменение логарифма дисперсии во времени}\\
\Delta\kappa &\sim \mathcal{GP}(\psi_\mathcal{GP}); f:\Real \to \Real&\text{изменение статистики корреляции во времени}\\
\end{align*}
\begin{align}
\mu_t &= \begin{cases}
\mu_\text{is} + \Delta\mu(t), t \in \text{\{обучающий период\}}\\
\mu_\text{oos} + \Delta\mu(t), t \in \text{\{тестовый период\}}
\end{cases}&\text{среднее доходностей в момент t}\nonumber\\
\sigma^2_t &= \exp(\nu + \Delta\nu(t))& \text{дисперсия доходностей в момент t}\nonumber\\
\rho_t &= \text{sigmoid}(\kappa + \Delta\kappa(t))&\text{корреляция доходностей в момент t}\nonumber\\
\Sigma_t &= ((1-\rho_t)\mathbb{I} + \rho_t) \odot \sigma_t\sigma_t^\top&\text{ковариационная матрица в момент t}\nonumber\\
R_t &\sim \mathcal{N}(\mu_t,\Sigma_t)& \text{наблюдаемые доходности в момент t}\label{eq:dyncorr}
\end{align}
где $k$ -- количество алгоритмов. Априорные распределения задаются исходя из природы данных и не могут быть указаны на этом этапе.