\chapter*{Заключение}
\addcontentsline{toc}{chapter}{Заключение}

В ходе проведенного исследования решены следующие задачи:
\begin{enumerate}
	\item Предложена общая схема моделирования динамики торговых стратегий, которая учитывает особенности, связанные с их доходностями. Результаты исследования включают в себя реализацию трех спецификаций модели с учетом, без учета динамики корреляций и упрощенная -- без корреляций. Для оценки модели динамики корреляций предложена модификация базового метода DECO с использованием гауссовского процесса

	\item Все три спецификации модели динамики доходностей торговых стратегий реализованы на языке \texttt{Python} с использованием специализированных библиотек для байесовского моделирования
	
	\item Реализована процедура составления портфеля торговых стратегий методом монте--карло

	\item Проведенный эмпирический анализ позволил сделать вывод о непостоянстве волатильности и постоянстве корреляций доходностей торговых стратегий

	\item С помощью сравнения методом бутстрап показана практическая значимость предложенного подхода для использования его в качестве инструмента для составления портфеля торговых стратегий. Портфель, основанный на предложенной модели оказывается эффективнее модели Марковица по ряду критериев (имеет большую обобщающую способность, более высокий коэффициент Шарпа на экзаменационной выборке). Предложенная позволяет получить более устойчивый во времени портфель.
\end{enumerate}

Проведенное исследование подтверждает целесообразность использования байесовских методов в формировании портфеля торговых стратегий. Такой подход позволяет учесть априорные знания о модели. Портфель, построенный на симуляциях из нее, получается более устойчивым во времени и, как следствие, более доходным на экзаменационном периоде.

