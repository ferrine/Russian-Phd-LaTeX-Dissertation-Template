\chapter{Проблема построения портфеля торговых стратегий на финансовом рынке}
Составление портфеля из торговых стратегий (алгоритмов\footnote{В работе понятия торговой стратегии и алгоритма -- синонимичны}) -- задача, которой стали активно заниматься крупные хедж-фонды. В отличии от финансовых активов, которые традиционно используются для составления портфеля, число алгоритмов, из которых можно составлять портфель, практически не ограничено. Причем алгоритмы рассматриваются как источник стохастического денежного потока, доходностей, что делает работу с ними неотличимой от работы с привычными финансовыми инструментами.

Разнообразие алгоритмов открывает возможности для диверсификации при составлении портфеля, однако, для этого необходима исследовательская работа. Свойства распределения доходностей алгоритмов не поддаются интерпретации, мотивация, которая стоит за идеей использования алгоритмов в качестве компонент портфеля -- возможность найти стабильные алгоритмы, которые минимально подвержены специфичным для компаний шокам, распределение доходности которых стационарно. Из стабильных алгоритмов можно составлять портфель с предсказуемыми свойствами. 

\section{Торговые стратегии и диверсификация рисков}
На рынке присутствует огромное количество активов с разными свойствами. Все активы в большей или меньшей степени подвержены рискам, например:
\begin{enumerate}
	\item рыночный риск -- непредсказуемое поведение рынка, влияющее на все или часть секторов
	\item риск ликвидности -- несоответствие ожидаемых и действительных сроков сделок
	\item риски компаний -- асимметрия информации инвестора и менеджмента компании
\end{enumerate}
Инвестора будем рассматривать как рационального агента, который имеет некоторую функцию предпочтений относительно активов, так как целью является практическое применение модели для формирование портфеля. 

\note{Из нескольких активов можно комбинировать портфель с приемлемым соотношением риск-доходность}
Дискретный выбор одного единственного актива (как и алгоритма) может не быть оптимальным выбором с точки зрения инвестора. Поскольку разные активы имеют разное соотношение риска и доходности, имеет смысл выбирать тот взвешенный набор активов, который бы максимизировал заданную функцию предпочтений, диверсифицируя риск \citep{markovitz1959}. С математической точки зрения всевозможные выпуклые комбинации активов в портфеле образуют допустимое множество соотношений риск-доходность. Выбор оптимальной -- задача портфельной оптимизации. При этом, учитываются особенности совместного распределения доходностей активов.

\note{Про алгоритмы. Необходимо четко обозначить то, что такое алгоритм и что это набор заранее определенных правил, кем то придуманный. Не останавливаясь на том, как они создаются, это будет в следующей части. Цели создания алгоритмов}
Существуют различные работы подтверждающие факт непостоянства рыночной ситуации \citep{billio2003, koutmos2012}. Меняются ожидаемые доходности активов и риски связанные с каждым из них. Таким образом, может быть полезным периодически переоценивать ситуацию на рынке и менять структуру портфеля.

Заранее заданные правила пересмотра называют торговой стратегией (алгоритмом). В течении периода работы алгоритм составляет динамически меняющийся портфель. Есть надежда, что торговая стратегия позволит диверсифицировать риски, связанные с изменением рыночной конъюнктуры и, в тоже время, риски, связанные с отдельными компаниями \citep{lorenz2008thesis}.

\note{Надежды на возможность предсказать движение рыночных сил тут нет, лишь попытка снизить риск}
Автономность алгоритма и абстракция от специфики отрасли или отдельной компании дают надежду на то, что в какой-то степени получится этих рисков избежать. Стоит заметить, что возникают риски специфичные именно для алгоритмов:
\begin{itemize}
	\item риск непредсказуемого поведения
	\item риск смещенной оценки статистических показателей
\end{itemize}
Эти риски возникают по причине того, что каждый алгоритм создается человеком, наблюдающим результат процесса своей деятельности (создание алгоритма). Будущее поведение алгоритма не определено.

Отдельно стоит отметить, что не стоит цели обыграть рынок отдельным алгоритмом, напротив, важно разнообразие, которое получается благодаря бесконечному числу возможных торговых стратегий. Стратегии могут быть любые (возможно даже случайные), главное, чтобы они были не похожи друг на друга по поведению и реакции на шоки рынка. Это разнообразие особенно интересно с точки зрения портфельной теории, однако вопрос этот плохо изучен в литературе. Большинство работ посвящены анализу одной стратегии, но не портфелю из них \citep{lorenz2008thesis, nuti2011}. В этой работе будет предпринята попытка изучить эту сторону вопроса и попробовать учесть специфику алгоритмической торговли при составлении портфеля.

\section{Особенности распределения доходностей торговых стратегий}
\note{Структурные сдвиги}
Первая важная особенность торговых стратегий состоит в том, что они создаются человеком, автором. Автору доступна информация о том, как бы действовал алгоритм на исторических данных. Эта процедура называется <<бэктест>>, с ее помощью можно получить огромное количество статистик для дальнейшего анализа эффективности стратегии. Ориентируясь на эту информацию, можно составить стратегию, которая хорошо работает на исторических данных. 

К сожалению, на последующем периоде поведение алгоритма непредсказуемо. Это является структурным сдвигом для торговой стратегии. Все, что было до момента создания, принимать во внимание, конечно, стоит, но с большой осторожностью. Доверять следует статистикам, полученным на периоде после создания. В противном случае, получится ситуация, когда на новых данных наилучшая стратегия ведет себя исключительно плохо в будущем, и, что хуже всего, мы были бы уверены в ее надежности и качестве. 

Алгоритмов может быть бесконечно много, все их невозможно использовать для построения портфеля, есть необходимость выбирать, какие алгоритмы участвуют в оптимизации, а какие -- нет. Механизмы селекции алгоритмов добиваются того, чтобы поведения <<до>> и <<после>> были близки хотя бы приблизительно, доходности которых хотя бы стационарны. На практике таких алгоримов очень немного.
 
\note{Динамика волатильности (часто наблюдается и вообще у активов)}
\citep{dumas1998} рекомендует учитывать непостоянство волатильности при анализе финансовых рядов. Оснований утверждать, что для стратегий волатильность доходностей постоянна во времени -- нет. Корректная оценка рисков, связанных с непостоянной волатильностью, требует модели волатильности.

\note{Изменение характера взаимосвязи, уточнив., что гипотеза основана на анализе просто активов}
В ряде исследований \citep{vaga1990, oral2017} был выявлен факт непостоянства корреляций между доходностями компаний во времени. Так, например, в период кризисов корреляции усиливается. Рынок сильно влияет на эти связи непредсказуемым образом, похожие эффекты могут наблюдаться и для торговых стратегий. Моделирование динамики взаимосвязей поможет должным образом учесть эти риски.

\section{Основные подходы формирования портфеля торговых стратегий}
\note{Критика портфельной теории Mарковица в литературе}
Портфельная теория Марковица \citep{markovitz1959} -- большой прорыв в решении задачи портфельной оптимизации. Тем не менее, у нее есть ряд недостатков\citep{lorenz2008thesis}:
\begin{itemize}
	\item инвестирование однопериодное
	\item модель не учитывает особенности распределения, только первый и второй центральные моменты
	\item инвестор не меняет состав портфеля
\end{itemize}
\note{Что вместо марковица?}
\cite{lorenz2008thesis, bucciol2006} предлагают использовать идеи \cite{neumann1944} и максимизировать ожидаемую функцию полезности инвестора. Это позволяет обобщить портфельную теорию, оптимальный портфель Марковица выступает частным случаем при определенных предпосылках. Этот подход позволяет производить оптимизацию портфеля методом Монте--Карло.

\note{и финальная фраза о том, что нам нужен последний метод}
Последний подход позволит учесть особенности распределения торговых стратегий, он и будет использован для составления оптимального портфеля. Задачи, которые необходимо решить в рамках этого подхода:
\begin{enumerate}
	\item составить модель динамики доходностей торговых стратегий и учесть:
	\begin{enumerate}
		\item структурные сдвиги
		\item динамику волатильности
		\item динамику корреляций
	\end{enumerate}
	\item оценить вероятностную модель динамики доходностей
	\item сформировать портфель  используя метод Монте--Карло
\end{enumerate}